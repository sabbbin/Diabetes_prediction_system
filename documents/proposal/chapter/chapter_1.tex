     \chapter{Introduction}
        \pagenumbering{arabic}
        \section{Background Introduction}
       
            As motivated by the rapid growth of text data, text mining has been applied to discover
hidden knowledge from text in many applications and domains. In business sectors,
great efforts have been made to find out customers’ sentiments and opinions, often
expressed in free text, towards companies’ products and services. However, discovering
sentiments and opinions through manual analysis of a large volume of textual data is
extremely difficult. Hence, in recent years, there have been much interests in the natural
language processing community to develop novel text mining techniques with the
capability of accurately extracting customers’ opinions from large volumes of
unstructured text data. Among various opinion mining tasks, one of them is sentiment
classification, Sentiment analysis is a type of data mining that measures the inclination
of people’s opinions through natural language processing (NLP), computational
linguistics and text analysis, which are used to extract and analyze subjective
information from the web – mostly social media and similar sources
[1]
. The analyzed
data quantifies the general public’s sentiments or reactions toward certain products,
people or ideas and reveal the contextual polarity of the information. Sentiment analysis
is also known as opinion mining. Sentiment analysis carries the basic task of
classification of the expressed opinion in a document into “positive”, “neutral”,
“negative”.
        \section{Problem Statement}
            As motivated by the rapid growth of text data, text mining has been applied to discover
hidden knowledge from text in many applications and domains. In business sectors,
great efforts have been made to find out customers’ sentiments and opinions, often
expressed in free text, towards companies’ products and services. However, discovering
sentiments and opinions through manual analysis of a large volume of textual data is
extremely difficult. Hence, in recent years, there have been much interests in the natural
language processing community to develop novel text mining techniques with the
capability of accurately extracting customers’ opinions from large volumes of
unstructured text data.

Among various opinion mining tasks, one of them is sentiment classification, Sentiment
analysis is a type of data mining that measures the inclination of people’s opinions
through natural language processing (NLP), computational linguistics and text analysis,
which are used to extract and analyze subjective information from the web - mostly
social media and similar sources. The analyzed data quantifies the general public's
sentiments or reactions toward certain products, people or ideas and reveal the
contextual polarity of the information. Sentiment analysis is also known as opinion
mining. Sentiment analysis carries the basic task of classification of the expressed
opinion in a document into “positive”, “neutral”, “negative”.
In recent years, many Nepali online news portals like onlinekhabar.com, ratopati.com, 
setopati.com 
have been providing news online in Nepali Unicode. Similarly, people use
Nepali Unicode to comment, review such online news. Not much work about Nepali
Natural Language Processing has been found during our research. Since the Nepali
language is morphologically rich and complex the text classifier needs to consider
specific language features before classifying the text. Preprocessing of data in Nepali
Unicode is at a delicate stage. Very low resources about Nepali Unicode are available.
The applications for sentiment analysis are endless. More and more we’re seeing it
being used in social media monitoring and VOC to track customer reviews, survey
responses, competitors, etc. However, it is also practical for use in business analytics
and situations in which text needs to be analyzed. Sentiment analysis is mostly used to
create recommendation systems which are user specific. The opinion of a user about a
specific product is the key factor for determining the likes and dislikes of the user which
results in increase of efficient of traditional recommender systems. Today’s consumers
buy products recommended by Amazon, watch movies recommended by Netflix, and
listen to music recommended by Pandora. A research showed that the recommendation
system developed using sentiment analysis of user reviews which was used by Amazon,
increased their profit boost by 29 percent. Similar results can be achieved in the Nepali
market by using a sentiment analyzer for Nepali language.
       \section{Objective}
            The main aim of this project is:
            \begin{enumerate}
                 \item To test various models for sentiment classification.
                \item To design and implement a system which can classify sentiment of
Devanagari sentences.
            \end{enumerate}