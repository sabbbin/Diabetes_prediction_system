    \chapter{Literature Review}
       Previously a research paper [2]for sentiment analysis by Peter D. Turney started the
classification of sentiment of reviews as recommended (thumbs up) or not
recommended (thumbs down). This algorithm achieved an average accuracy of 74%.
Chandan Prasad Gupta and Bal Krishna Bal also proposed [3] the system of Detecting
Sentiment in Nepali Texts by using self-developed Nepali Sentiment Corpus and Nepali
SentiWordNet. Their system does not use Neural Networks to classify the Nepali texts.
The major breakthrough in this field happened when researchers at Stanford University
proposed the recurrent neural network system for noise reduction in ASR system
\cite{liu2012sentiment}
.
This gave a new approach to tackle sequential data in the field of natural language
processing and machine learning in general. The recurrent neural network has
outperformed different other models in this task of analyzing and processing sequential
data such as a sequence of text.
The paper \cite{shrestha2016new} by Mikolov et. al. discusses the efficient estimation of word embedding
using skip gram approach which can be considered as one of the groundbreaking works
increasing the efficiency of the classification system. Similarly, other various authors
have used various natural language processing techniques to classify sentiment of texts
in different languages. Like this paper
\cite{kaya2012sentiment}
the author discusses the approach taken to
analyze Turkish political news. Likewise, in this paper
\cite{sharma2015practical}
author has used a lexicon-
based approach for classifying Indian text which is lexically similar to Nepali.
Similarly, a latest research paper
\cite{vaswani2017attention}
by researchers at Google Brain studies about the
attention model for sequence learning tasks where model learns to put attention to
certain words than other for output similar to how the brain works.